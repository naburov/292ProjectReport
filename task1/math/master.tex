

\documentclass[../master.tex]{subfiles}
\begin{document}

\textbf{Теорема}. Асимптотическое решение задачи~1 имеет вид:
\begin{align*}
u &= 1 + u_0^{\text{II}}(x, \tau)  + 
 \varepsilon^{1/3}\big(u_1^{\text{I}}( \xi_1, \xi_2, \theta)  + u_1^{\text{II}}(  \xi_1, \xi_2, \tau)\big), + O(\varepsilon^{2/3}), \\
v&= \varepsilon^{2/3}\big(v_2^{\text{I}}(\xi_1, \xi_2, \theta) + v_2^{\text{II}}(\xi_1, \xi_2, \tau)\big) +  O(\varepsilon), \\
w &= \varepsilon^{1/3} w_1^{\text{I}}(  \xi_1, \xi_2, \theta) + O(\varepsilon^{2/3}), \\
p &= p_0 + \varepsilon^{2/3}p_2^{\text{II}}(\xi_1, \xi_2, \tau)+ O(\varepsilon),
\end{align*}
где $\displaystyle \xi_1 = \dfrac{x - x_0}{\varepsilon},\ 
\xi_2 = \dfrac{z - z_0}{\varepsilon}, \ 
\tau = \dfrac{y_w}{\varepsilon},\ \theta= \dfrac{y_w}{\varepsilon^{\frac{4}{3}}}$,

$y_w = y-y_s$~--- переменная, выравнивающая границу,

$\tau$ и $\theta$~--- погранслойные переменные для II и I палуб соответственно. 

 Функция $u_0^{\text{II}} = u^*-1$, $u_1^\mathrm{II} = \mu \dfrac{\partial u_0^{\text{II}}}{\partial \tau}\bigg|_{ x=x_0 }$, где
функция $u^* =  f^{\prime}\bigg(\dfrac{\tau}{\sqrt{x}}\bigg) $, 

$f(\gamma)$~--- функция Блазиуса,  являющаяся решением следующей краевой задачи: 
\[ f^{\prime \prime \prime} = -\frac{1}{2}f f^{\prime \prime}, \quad 
f(0) = f^{\prime}(0) = 0, \
f^{\prime}(\infty) = 1.
\]

Функции $u_1^{\text{I}}$, $v_2^{\text{I}}$, $w_1^{\text{I}}$ определяются из соотношений:
\[
 u^{\dagger} = u_1^{\text{I}} + u_1^{\text{II}}\big|_{ \tau=0} + \theta\dfrac{\partial u_0^{\text{II}}}{\partial \tau}\bigg|_{{\begin{smallmatrix} \tau=0 \\ x=x_0 \end{smallmatrix}}}, \
 v^{\dagger} = v_2^{\text{I}} + v_2^{\text{II}}\big|_{\tau=0}, \
 w^{\dagger} = w_1^{\text{I}},
\]
где функции $u^\dagger$, $v^\dagger$, $w^\dagger$ являются решением краевой задачи для системы уравнений Прандтля с самоиндуцированным давлением:
\begin{equation*}
\begin{cases}
u^{\dagger}\bigg(\dfrac{\partial u^{\dagger}}{\partial\xi_1} - \dfrac{\partial \mu}{\partial\xi_1}\dfrac{\partial u^{\dagger}}{\partial\theta} \bigg)+ w^{\dagger}\bigg(\dfrac{\partial u^{\dagger}}{\partial\xi_2} - \dfrac{\partial \mu}{\partial\xi_2}\dfrac{\partial u^{\dagger}}{\partial\theta} \bigg) + \\ 
\quad \hfill +v^{\dagger}\dfrac{\partial u^{\dagger}}{\partial\theta} + \dfrac{\partial p_2^{\text{II}}}{\partial \xi_1}\bigg|_{\tau=0} -  \dfrac{\partial^2u^{\dagger}}{\partial\theta^2} =0, 
\\[2ex]
 v^{\dagger}\dfrac{\partial w^{\dagger}}{\partial\theta} + w^{\dagger} \bigg(\dfrac{\partial w^{\dagger}}{\partial\xi_2} - \dfrac{\partial \mu}{\partial\xi_2}\dfrac{\partial w^{\dagger}}{\partial\theta} \bigg) + u^{\dagger} \bigg(\dfrac{\partial w^{\dagger}}{\partial\xi_1} - \dfrac{\partial \mu}{\partial\xi_1}\dfrac{\partial w^{\dagger}}{\partial\theta} \bigg)-\\ \quad \hfill - \dfrac{\partial^2 w^\dagger}{\partial \theta^2}= 0,
 \\[2ex]
\dfrac{\partial u^{\dagger}}{\partial\xi_1} - \dfrac{\partial \mu}{\partial\xi_1}\dfrac{\partial u^{\dagger}}{\partial\theta} + \dfrac{\partial w^{\dagger}}{\partial\xi_2} - \dfrac{\partial \mu}{\partial\xi_2}\dfrac{\partial w^{\dagger}}{\partial\theta} + \dfrac{\partial v^{\dagger}}{\partial \theta} = 0,
\end{cases}
\end{equation*}
\begin{align*}
&u^{\dagger} \big|_{\theta=0} = 0,\
v^{\dagger} \big|_{\theta=0} = 0, \
w^{\dagger} \big|_{\theta=0} = 0,\\
&\dfrac{\partial u^{\dagger}}{\partial \theta}\bigg|_{\theta \rightarrow \infty} = \dfrac{\partial u_0^{\text{II}}}{\partial \tau}\bigg|_{\begin{smallmatrix} \tau=0\\ x=x_0   \end{smallmatrix}}, \
w^{\dagger} \big|_{\theta\to \infty} =0,\\
&u^{\dagger} \big|_{\xi_{1,2} \rightarrow \pm \infty} = \theta \dfrac{\partial u_0^{\text{II}}}{\partial \tau}\bigg|_{{\begin{smallmatrix} \tau=0\\ x=x_0  \end{smallmatrix}}}, \
v^{\dagger} \big|_{\xi_{1,2} \rightarrow \pm \infty} =0,\
w^{\dagger} \big|_{\xi_{1,2} \rightarrow \pm \infty} =0.
\end{align*}

Функция $v_2^{\text{II}}$ является решением краевой задачи для уравнения типа Рэлея: ($\displaystyle \Delta_{\xi_1, \xi_2, \tau} = \frac{\partial^2}{\partial  \xi_1^2} + \frac{\partial^2}{\partial \xi_2^2} + \frac{\partial^2}{\partial \tau^2}$):
\[ u^*\big|_{\begin{smallmatrix} \\ x=x_0  \end{smallmatrix}}\Delta_{\xi_1, \xi_2, \tau}v_2^{\text{II}} - v_2^{\text{II}}\dfrac{\partial^2 u^*}{\partial \tau^2}\bigg|_{\begin{smallmatrix} \tau=0\\ x=x_0 \end{smallmatrix}} = 0\]
\[ v_2^{\text{II}} \big|_{\xi_{1,2} \rightarrow \pm \infty} = 0, \
 v_2^{\text{II}} \big|_{\tau \rightarrow \infty} = 0, \
v_2^{\text{II}} \big|_{\tau =0} = v^{\dagger}\big|_{\theta \rightarrow \infty}.\]


Давление $p_2^{\text{II}}$  определяется равенством:
\[
p_2^\mathrm{II} = \int\limits_{-\infty}^{\xi_2} \frac{\partial w_2^\mathrm{II}}{\partial \xi_1}\,d\xi_2',
\]
где функция $w_2^{\text{II}}$ является решением краевой задачи для уравнения Пуассона ($\Delta_{\xi_1, \xi_2} = \dfrac{\partial^2}{\partial  \xi_1^2} + \dfrac{\partial^2}{\partial \xi_2^2}$):
\[
u^*\big|_{ x=x_0} \Delta_{\xi_1, \xi_2}w_2^{\text{II}} = \dfrac{\partial v_2^{\text{II}}}{\partial \xi_2}\dfrac{\partial u^*}{\partial \tau}\bigg|_{x=x_0}\!\!\! - \dfrac{\partial^2 v_2^{\text{II}}}{\partial \xi_2 \partial\tau}u^*\big|_{x=x_0 },\
w_2^{\text{II}} \big|_{\xi_{1,2} \rightarrow \pm \infty} = 0.
\]
 Выражение для самоиндуцированного давления можно записать в виде:
\[
\dfrac{\partial p_2^{\text{II}}}{\partial \xi_1}\bigg|_{\tau=0} =- v^\dagger\big|_{\theta\to\infty} \dfrac{\partial u_0^{\text{II}}}{\partial \tau}\bigg|_{\begin{smallmatrix}
   \tau=0 \\ x=x_0 \end{smallmatrix}}.
\]

\textbf{Доказательство}. Согласно теории двухпалубного пограничного слоя можно выделить три зоны потока, набегающего на поверхность: $\text{I}, \text{II}, \text{EXT}$. Тогда изначальное асимптотическое разложение для компонент скорости $u, v, w$, а также давления $p$ будет иметь вид:
\begin{equation}
  \begin{gathered}[b]
	u = 1 + u_0^{\text{II}} + \varepsilon^{1/3}(u_1^{\text{EXT}}  +  u_1^{\text{II}} + u_1^{\text{I}}) + \varepsilon^{2/3} (u_2^{\text{EXT}} +  u_2^{\text{II}} + u_2^{\text{I}}) + \varepsilon^{3/3} (u_3^{\text{EXT}} +  u_3^{\text{II}} + u_3^{\text{I}}) \\
	v = \varepsilon^{2/3} (v_2^{\text{EXT}} +  v_2^{\text{II}} + v_2^{\text{I}}) + \varepsilon^{3/3} (v_3^{\text{EXT}} +  v_3^{\text{II}} + v_3^{\text{I}}) \\
	w = \varepsilon^{1/3}(w_1^{\text{EXT}}  +  w_1^{\text{II}} + w_1^{\text{I}}) + \varepsilon^{2/3} (w_2^{\text{EXT}} +  w_2^{\text{II}} + w_2^{\text{I}}) + \varepsilon^{3/3} (w_3^{\text{EXT}} +  w_3^{\text{II}} + w_3^{\text{I}}) \\
	p = p_0 + \varepsilon^{1/3}(p_1^{\text{EXT}}  +  p_1^{\text{II}} + p_1^{\text{I}}) + \varepsilon^{2/3} (p_2^{\text{EXT}} +  p_2^{\text{II}} + p_2^{\text{I}}) + \varepsilon^{3/3} (p_3^{\text{EXT}} +  p_3^{\text{II}} + p_3^{\text{I}})
  \end{gathered}
\end{equation}

Эти выражения для $u, v, w, p$ необходимо подставить в систему уравнений Навье-Стокса.
Для того, чтобы получить уравнения в области EXT, необходимо устремить погранслойные переменные $\theta$ и $\tau$ к бесконечности, и сгруппировать слагаемые при одинаковых степенях при $\varepsilon$. При устремлении $\theta$ к бесконечности все функции c верхним индексом I будут стремиться к 0, а при устремлении $\tau$ к бесконечности к 0 будут стремиться все функции с верхним индексом II.
При $\varepsilon=-2/3$ получим следующую систему уравнений:
\begin{equation}
	\begin{cases}
	\dfrac{\partial p_1^{\text{EXT}}}{\partial \xi_1} + \dfrac{\partial u_1^{\text{EXT}}}{\partial \xi_1} = 0 \\[2ex]
	\dfrac{\partial p_1^{\text{EXT}}}{\partial \xi_2} + \dfrac{\partial w_1^{\text{EXT}}}{\partial \xi_1} = 0 \\[2ex]
	\dfrac{\partial w_1^{\text{EXT}}}{\partial \xi_2} + \dfrac{\partial u_1^{\text{EXT}}}{\partial \xi_1} = 0
	\end{cases}
\end{equation}

Для того, чтобы получить уравнения в области II, необходимо устремить погранслойные переменную $\theta$ к бесконечности, и сгруппировать слагаемые при одинаковых степенях при $\varepsilon$. 
При $\varepsilon=-1$ получим следующую систему уравнений:
\begin{equation}
	\begin{cases}
	\dfrac{\partial u_0^{\text{II}}}{\partial \xi_1} + u_0^{\text{II}}\dfrac{\partial u_0^{\text{II}}}{\partial \xi_1} = 0 \\[2ex]
	\dfrac{\partial u_0^{\text{II}}}{\partial \xi_1} = 0 \\[2ex]
	\end{cases}
\end{equation}

При $\varepsilon=-2/3$ получим следующую систему уравнений:
\begin{equation}
	\begin{cases}
	u^*\bigg(\dfrac{\partial u_1^{\text{II}}}{\partial \xi_1} - \dfrac{\partial \mu}{\partial \xi_1} \dfrac{\partial u_0^{\text{II}}}{\partial \tau} \bigg) = 0 \\[2ex]
	u^* \dfrac{\partial w_1^{\text{II}}}{\partial \xi_1} = 0 \\[2ex]
	- \dfrac{\partial \mu}{\partial \xi_1} \dfrac{\partial u_0^{\text{II}}}{\partial \tau} + \dfrac{\partial u_1^{\text{II}}}{\partial \xi_1} + \dfrac{\partial w_1^{\text{II}}}{\partial \xi_2} = 0
	\end{cases}
\end{equation}

При $\varepsilon=-1/3$ получим следующую систему уравнений:
\begin{equation}
	\begin{cases}
		u^*\bigg(\dfrac{\partial u_2^{\text{II}}}{\partial \xi_1} - \dfrac{\partial \mu}{\partial \xi_1} \dfrac{\partial u_1^{\text{II}}}{\partial \tau} \bigg)  + v_2^{\text{II}}\dfrac{\partial u_0^{\text{II}}}{\partial \tau} + \dfrac{\partial p_2^{\text{II}}}{\partial \xi_1}= 0 \\[2ex]
		\dfrac{\partial p_2^{\text{II}}}{\partial \tau} + \dfrac{\partial v_2^{\text{II}}}{\partial \xi_1}u^* = 0  \\[2ex]
		\dfrac{\partial p_2^{\text{II}}}{\partial \xi_1} + \dfrac{\partial w_2^{\text{II}}}{\partial \xi_1}u^* = 0 \\[2ex]
		-\dfrac{\partial \mu}{\partial \xi_1} \dfrac{\partial u_1^{\text{II}}}{\partial \tau} +  \dfrac{\partial v_2^{\text{II}}}{\partial \tau} + \dfrac{\partial w_2^{\text{II}}}{\partial \xi_2} +  \dfrac{\partial u_2^{\text{II}}}{\partial \xi_1} = 0
	\end{cases}
\end{equation}

Для того, чтобы получить уравнения в области I, необходимо сгруппировать слагаемые при одинаковых степенях при $\varepsilon$. 
При $\varepsilon=-1$ получим, что:
\begin{equation}
	\dfrac{\partial p_1^{\text{I}}}{\partial \theta} = 0
\end{equation}

При $\varepsilon=-2/3$ получим следующую систему:
\begin{equation}
	\begin{cases}
		u^*\dfrac{\partial u_1^{\text{I}}}{\partial \xi_1} -u^* \dfrac{\partial \mu}{\partial \xi_1} \dfrac{\partial u_1^{\text{I}}}{\partial \theta}  + w_1^{\text{I}}\dfrac{\partial u_0^{\text{II}}}{\partial \xi_2} +  u_1^{\text{I}}\dfrac{\partial u_0^{\text{II}}}{\partial \xi_1} = 0 \\[2ex]		
		\dfrac{\partial p_2^{\text{I}}}{\partial \theta} = 0 \\[2ex]
		u^*\bigg( \dfrac{\partial w_1^{\text{I}}}{\partial \xi_1} -  \dfrac{\partial \mu}{\partial \xi_1} \dfrac{\partial w_1^{\text{I}}}{\partial \theta} \bigg) = 0 \\[2ex]
		-\dfrac{\partial \mu}{\partial \xi_1} \dfrac{\partial u_1^{\text{I}}}{\partial \theta} + 
			 \dfrac{\partial v_2^{\text{I}}}{\partial \theta} - 
			 \dfrac{\partial \mu}{\partial \xi_2} \dfrac{\partial w_1^{\text{I}}}{\partial \theta} + 
			 \dfrac{\partial u_1^{\text{I}}}{\partial \xi_1} + 
			 \dfrac{\partial w_1^{\text{I}}}{\partial \xi_2} = 0
	\end{cases}
\end{equation}

При $\varepsilon=-1/3$ получим следующую систему:
\begin{equation}
	\begin{cases}
		-\dfrac{\partial \mu}{\partial \xi_2} \dfrac{\partial u_2^{\text{I}}}{\partial \theta} -\dfrac{\partial \mu}{\partial \xi_1} \dfrac{\partial u_2^{\text{I}}}{\partial \theta} + 
		\dfrac{\partial v_3^{\text{II}}}{\partial \theta} + \dfrac{\partial u_2^{\text{I}}}{\partial \xi_2} + \dfrac{\partial u_2^{\text{I}}}{\partial \xi_1} = 0 \\[2ex]
		\dfrac{p_3^{\text{I}}}{\partial \theta} + u^*\bigg(\dfrac{\partial v_2^{\text{I}}}{\partial \xi_1} - \dfrac{\partial \mu}{\partial \xi_1} \dfrac{\partial v_2^{\text{I}}}{\partial \theta} \bigg )= 0 \\[2ex]
		u^*\bigg( \dfrac{\partial u_2^{\text{I}}}{\partial \xi_1} - \dfrac{\partial \mu}{\partial \xi_1}\dfrac{\partial u_2^{\text{I}}}{\partial \theta} \bigg) +
			(u_1^{\text{I}} + u_1^{\text{II}})\bigg(  \dfrac{\partial w_1^{\text{I}}}{\partial \xi_1} - \dfrac{\partial \mu}{\partial \xi_1}\dfrac{\partial w_1^{\text{I}}}{\partial \theta} \bigg) +
			(v_2^{\text{I}} + v_2^{\text{II}})\dfrac{\partial w_1^{\text{I}}}{\partial \theta} + \\[2ex] \quad \hfill +
			(w_1^{\text{I}} + w_1^{\text{II}})\bigg(  \dfrac{\partial w_1^{\text{I}}}{\partial \xi_1} - \dfrac{\partial \mu}{\partial \xi_1}\dfrac{\partial w_1^{\text{I}}}{\partial \theta} \bigg) -
			\dfrac{\partial^2 w_1^{\text{I}}}{\partial \theta^2} = 0 \\[2ex]
		-\dfrac{\partial \mu}{\partial \xi_2} \dfrac{\partial u_2^{\text{I}}}{\partial \theta} + 
			 \dfrac{\partial v_3^{\text{I}}}{\partial \theta} - 
			 \dfrac{\partial \mu}{\partial \xi_1} \dfrac{\partial u_2^{\text{I}}}{\partial \theta} + 
			 \dfrac{\partial u_2^{\text{I}}}{\partial \xi_2} + 
			 \dfrac{\partial u_2^{\text{I}}}{\partial \xi_1} = 0
	\end{cases}
\end{equation}

Разложим функции с верхним индексом II в асимтотический ряд и заменим их в системах уравений на слой I: 
\begin{align*}
	f(..., \tau) = f\big|_{\tau=0} + \epsilon\tau\dfrac{\partial f}{\partial \tau}\bigg|_{\tau = 0} + O(\varepsilon^{2})& = \\
	=  f\big|_{\tau=0} + \epsilon^{1/3}\theta\dfrac{\partial f}{\partial \tau}\bigg|_{\tau = 0} + O(\varepsilon^{2/3})
\end{align*}

Получим для системы при $\varepsilon=-2/3$:
\begin{equation}
	\begin{cases}
		u^*\big|_{\tau=0}\bigg( \dfrac{\partial u_1^{\text{I}}}{\partial \xi_1} - \dfrac{\partial \mu}{\partial \xi_1} \dfrac{\partial u_1^{\text{I}}}{\partial \theta} \bigg) = 0 \\[2ex]
		\dfrac{\partial p_2^{\text{I}}}{\partial \theta} = 0 \\[2ex]
		u^*\big|_{\tau=0}\bigg( \dfrac{\partial w_1^{\text{I}}}{\partial \xi_1} -  \dfrac{\partial \mu}{\partial \xi_1} \dfrac{\partial w_1^{\text{I}}}{\partial \theta} \bigg) = 0 \\[2ex]
		-\dfrac{\partial \mu}{\partial \xi_1} \dfrac{\partial u_1^{\text{I}}}{\partial \theta} + 
			 \dfrac{\partial v_2^{\text{I}}}{\partial \theta} - 
			 \dfrac{\partial \mu}{\partial \xi_2} \dfrac{\partial w_1^{\text{I}}}{\partial \theta} + 
			 \dfrac{\partial u_1^{\text{I}}}{\partial \xi_1} + 
			 \dfrac{\partial w_1^{\text{I}}}{\partial \xi_2} = 0
	\end{cases}
\end{equation}

Получим для системы при $\varepsilon=-1/3$:
\begin{equation}
	\begin{cases}
		\bigg(w_1^{\text{I}} + w_1^{\text{II}}\big|_{\tau=0}\bigg) \bigg(\dfrac{\partial u_1^{\text{I}}}{\partial \xi_2}
		- \dfrac{\partial \mu}{\partial \xi_2}\dfrac{\partial u_1^{\text{I}}}{\partial \theta} \bigg) + v_2^{\text{II}} \bigg(
		\dfrac{\partial u^*}{\partial \tau}\bigg|_{\tau=0} + \dfrac{\partial u_1^{\text{I}}}{\partial \theta} \bigg)+
		\\[2ex] \quad \hfill  u_1^{\text{I}}\bigg(\dfrac{\partial u_1^{\text{I}}}{\partial \xi_1} + \dfrac{\partial u_1^{\text{II}}}{\partial \xi_1}\bigg|_{\tau=0} \bigg) - \dfrac{\partial \mu}{\partial \xi_1}\dfrac{\partial u^*}{\partial \tau}\bigg|_{\tau=0}(w_1^{\text{I}} + u_1^{\text{I}}) + v_2^{\text{II}}\big|_{\tau=0}\dfrac{\partial u_1^{\text{I}}}{\partial \theta} +  \\[2ex] \quad \hfill + w_1^{\text{I}}\dfrac{\partial u_1^{\text{II}}}{\partial \xi_2}\bigg|_{\tau=0} + \theta \dfrac{\partial u^*}{\partial \tau}\bigg|_{\tau=0}\bigg(\dfrac{\partial u_1^{\text{I}}}{\partial \xi_1} - \dfrac{\partial \mu}{\partial \xi_1}\dfrac{\partial u_1^{\text{I}}}{\partial \theta} \bigg) - \dfrac{\partial^2 u_1^{\text{I}}}{\partial \theta^2} = 0 \\[2ex]
		\dfrac{\partial p_3^{\text{I}}}{\partial \theta} = 0 \\[2ex]
		\bigg(u_1^{\text{I}} + u_1^{\text{II}}\big|_{\tau=0}\bigg)\bigg(\dfrac{\partial w_1^{\text{I}}}{\partial \xi_1}
		- \dfrac{\partial \mu}{\partial \xi_1}\dfrac{\partial w_1^{\text{I}}}{\partial \theta} \bigg) + v_2^{\text{I}} \dfrac{\partial w_1^{\text{I}}}{\partial \theta} + \\[2ex] \quad \hfill + v_2^{\text{II}}\bigg|_{\tau=0}\dfrac{\partial w_1^{\text{I}}}{\partial \theta} + \bigg(w_1^{\text{I}} + w_1^{\text{II}}\big|_{\tau=0}\bigg) \bigg(\dfrac{\partial w_1^{\text{I}}}{\partial \xi_2} - \dfrac{\partial \mu}{\partial \xi_2}\dfrac{\partial w_1^{\text{I}}}{\partial \theta} \bigg) + \\[2ex] \quad \hfill + \theta \dfrac{\partial u^*}{\partial \tau}\bigg|_{\tau=0} \bigg(\dfrac{\partial w_1^{\text{I}}}{\partial \xi_1}	- \dfrac{\partial \mu}{\partial \xi_1}\dfrac{\partial w_1^{\text{I}}}{\partial \theta} \bigg) - \dfrac{\partial^2 w_1^{\text{I}}}{\partial \theta^2} = 0 \\[2ex]
		-\dfrac{\partial \mu}{\partial \xi_2} \dfrac{\partial u_2^{\text{I}}}{\partial \theta} + 
			 \dfrac{\partial v_3^{\text{I}}}{\partial \theta} - 
			 \dfrac{\partial \mu}{\partial \xi_1} \dfrac{\partial u_2^{\text{I}}}{\partial \theta} + 
			 \dfrac{\partial u_2^{\text{I}}}{\partial \xi_2} + 
			 \dfrac{\partial u_2^{\text{I}}}{\partial \xi_1} = 0		
	\end{cases}
\end{equation}

Сделаем следующие замены в системах уравений при  $\varepsilon=-1/3$ и  $\varepsilon=-2/3$:
\[
 u^{\dagger} = u_1^{\text{I}} + u_1^{\text{II}}\big|_{ \tau=0} + \theta\dfrac{\partial u_0^{\text{II}}}{\partial \tau}\bigg|_{{\begin{smallmatrix} \tau=0 \\ x=x_0 \end{smallmatrix}}}, \
 v^{\dagger} = v_2^{\text{I}} + v_2^{\text{II}}\big|_{\tau=0}, \
 w^{\dagger} = w_1^{\text{I}},
\]

И получим: 
\begin{equation*}
	\begin{cases}
		u^{\dagger}\bigg(\dfrac{\partial u^{\dagger}}{\partial\xi_1} - \dfrac{\partial \mu}{\partial\xi_1}\dfrac{\partial u^{\dagger}}{\partial\theta} \bigg)+ w^{\dagger}\bigg(\dfrac{\partial u^{\dagger}}{\partial\xi_2} - \dfrac{\partial \mu}{\partial\xi_2}\dfrac{\partial u^{\dagger}}{\partial\theta} \bigg) + \\ 
		\quad \hfill +v^{\dagger}\dfrac{\partial u^{\dagger}}{\partial\theta} + \dfrac{\partial p_2^{\text{II}}}{\partial \xi_1}\bigg|_{\tau=0} -  \dfrac{\partial^2u^{\dagger}}{\partial\theta^2} =0, 
		\\[2ex]
		 v^{\dagger}\dfrac{\partial w^{\dagger}}{\partial\theta} + w^{\dagger} \bigg(\dfrac{\partial w^{\dagger}}{\partial\xi_2} - \dfrac{\partial \mu}{\partial\xi_2}\dfrac{\partial w^{\dagger}}{\partial\theta} \bigg) + u^{\dagger} \bigg(\dfrac{\partial w^{\dagger}}{\partial\xi_1} - \dfrac{\partial \mu}{\partial\xi_1}\dfrac{\partial w^{\dagger}}{\partial\theta} \bigg)-\\ \quad \hfill - \dfrac{\partial^2 w^\dagger}{\partial \theta^2}= 0,
		 \\[2ex]
		\dfrac{\partial u^{\dagger}}{\partial\xi_1} - \dfrac{\partial \mu}{\partial\xi_1}\dfrac{\partial u^{\dagger}}{\partial\theta} + \dfrac{\partial w^{\dagger}}{\partial\xi_2} - \dfrac{\partial \mu}{\partial\xi_2}\dfrac{\partial w^{\dagger}}{\partial\theta} + \dfrac{\partial v^{\dagger}}{\partial \theta} = 0,
	\end{cases}
\end{equation*}


\end{document}