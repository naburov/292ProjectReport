\documentclass[../master.tex]{subfiles}
\begin{document}

 Для задач обтекания жидкостями и газами поверхностей с малыми неровностями при больших значениях числа Рейнольдса $\mathrm{Re}$ широко известны асимптотические решения с двух- и трехпалубными структурами пограничного слоя. 

Такие модели позволяют \emph{эффективно} исследовать\linebreak подобные задачи, не прибегая к ресурсоемким (из-за наличия пространственной разномасштабности)  методам  прямого численного моделирования уравнений Навье--Стокса.
проекта является обобщение двухпалубной структуры на ранее не исследованный класс задач:
\begin{enumerate}
    \item Обтекание трехмерной неровности на пластине; 
    \item Обтекание двумерной неровности на клине.
\end{enumerate}
Математическая модель задачи основывается на системе уравнений Навье--Стокса и неразрывности
\[\langle \mathbf{U}, \nabla \rangle \mathbf{U} = - \nabla p + \varepsilon^2 \Delta \mathbf{U}, \
\langle \nabla, \mathbf{U} \rangle = 0,\   
\]
 с граничными условиями прилипания к обтекаемой поверхности $y_s$, $\mathbf{U}\big|_{y=y_s} = 0$,  и согласованием с набегающим потоком $\mathbf u_\infty$ вдали от нее, $ \mathbf{U}\big|_{ x \to -\infty} = \mathbf u_\infty$.
 
Обозначения: $\mathbf U$~--- вектор скорости ($\mathbf U = (u,v,w)$ в трехмерном случае, и $\mathbf U = (u,v)$ в двумерном), $p$ --- давление.

\end{document}